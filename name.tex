\documentclass[a4paper,12pt]{article}
\usepackage[utf8]{inputenc} 
\usepackage[brazil]{babel} 
\usepackage[lmargin=3cm,tmargin=3cm,rmargin=2cm,bmargin=2cm]{geometry} 
\usepackage{amsmath,amsthm,amsfonts,amssymb,dsfont,mathtools} 
\usepackage{graphicx}
\usepackage{url}




\begin{document}
\title{Atividade de Matemática}
\date{}
\author{Marcelo Henrique Santana Sopa\\
RA: 0051352211012}
\maketitle

\[\]\

\textbf{Descrição da Base de Dados:}


A base de dados utilizada neste estudo é composta por registros de de alguns jogos considerados como os "melhores" desde 1998 à 2022. A classe "Title" comenta um pouco do que se trata o jogo em específico. A classe "Plataform" se refere em que tipo de consoles esses jogos poderiam ser jogados e a classe "Metascore" se refere a uma pontuação da avaliação desses jogos e a classe "Date" se refere a data em que esses jogos foram distribuídos.

\[\]\

\textbf{Resultados Entropia}\\
Com base nas classes \textit{"Date"}, \textit{"Platform"} e \textit{"Metascore"}

• Entropia da "Date": "6.33..."

• Entropia da Plataform: "3.74"

• Entropia do metascore: "2.20..."

\[\]\

\textbf{Entropia máxima}\\
Um valor de entropia máxima resultante de 7.9478502014851 indica que o conjunto de dados possui uma alta incerteza ou desordem. Isso significa que as classes estão distribuídas de forma relativamente uniforme e não há uma estrutura clara ou padrões distintivos nos dados.

Uma entropia máxima próxima de 8 sugere que as classes estão igualmente representadas no conjunto de dados. Isso pode ser interpretado como uma situação em que nenhuma classe é predominante em relação às outras. Em outras palavras, o conjunto de dados apresenta uma diversidade significativa ou falta de estrutura, o que pode dificultar a distinção ou separação das classes com base em suas características.

\newpage

\textbf{Fontes:}

Kaggle (url do dataset no repositório do GitHub).

\[\]\

\textbf{Conclusão:}
Com base na análise da entropia das variáveis do dataset de jogos, podemos chegar às seguintes conclusões:

A entropia da variável "Date" apresenta um valor alto de "6.33...", indicando uma distribuição heterogênea dos jogos ao longo do período mencionado. Isso sugere que os jogos foram lançados em datas variadas, sem uma concentração significativa em intervalos específicos.

A entropia da variável "Plataforma" possui um valor de "3.74", o que sugere uma distribuição mais concentrada dos jogos entre as diferentes plataformas. Isso indica que existe uma separação relativamente clara entre as opções de console, podendo haver uma preferência ou predominância de jogos em determinadas plataformas específicas.

A entropia da variável "Metascore" possui um valor de "2.20...", o que indica uma menor incerteza ou desordem nas avaliações das pessoas. Isso sugere que as avaliações estão mais concentradas, com uma separação relativamente clara entre as diferentes pontuações. Pode indicar uma tendência de avaliações mais polarizadas ou menos variabilidade nas avaliações dos jogos.

Essas conclusões fornecem uma visão inicial da distribuição temporal dos jogos, da preferência por plataformas específicas e da variabilidade nas avaliações dos jogos no dataset. No entanto, é recomendado realizar uma análise mais aprofundada considerando o contexto específico do problema e explorando outras variáveis relevantes para obter uma compreensão mais completa dos padrões e tendências presentes no dataset de jogos.















\end{document}